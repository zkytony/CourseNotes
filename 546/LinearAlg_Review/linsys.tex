\begin{definition}[Row Echelon Form]
Each variable can be the leading variable for at most one equation.
\end{definition}
For example, 
\begin{equation}
    \begin{aligned}
        x_1 + x_2 + x_3 - x_4 &=0\\
        -x_2 + 7x_4 - x_5 &=-1\\
        x_4+x_5&=2
    \end{aligned}
\end{equation}

\begin{definition}
Linear systems are \emph{equivalent} if they are related by a sequence of elementary operations:
    \begin{enumerate}[label={(\theenumi)}]
        \item Interchange position of rows
        \item Multiply an equal constant
        \item Add a multiple of one equation to another
    \end{enumerate}
\end{definition}

\begin{definition}[Augmented Matrix]
The linear system
\begin{equation}
\label{eq:lk}
    \begin{aligned}
        a_{11}x_1+a_{12}x_2+&\cdots+a_{1m}x_m= b_1,\\
        &\mathrel{\makebox[\widthof{=}]{\vdots}}  &\\
        a_{n1}x_1+a_{n2}x_2+&\cdots+a_{nm}x_m= b_n
    \end{aligned}
\end{equation}
can be written as an \emph{augmented matrix} as follows:
    \begin{equation}
        \begin{bmatrix}
        a_{11} & \dots & a_{1m} & b_1 \\
        \vdots & \ddots & \vdots & \vdots \\
        a_{n1} & \dots & a_{nm} & b_n \\
        \end{bmatrix}
    \end{equation}
\end{definition}

\begin{definition}[Row Echelon Form]
    A matrix is in \emph{row echelon form} if
    \begin{enumerate}[label=\alph*)]
        \item Every leading term is in a column to the left of the leading term of the row below it.
        \item Any zero rows are at the bottom of the matrix
    \end{enumerate}
\end{definition}
For example, the left matrix below is not an echelon form, because ``0=7'' has no leading variable. It is an \emph{inconsistent} matrix. The right matrix is a echelon form.
\begin{center}
    \begin{tabular}{c c}
        $
        \begin{bmatrix}
        1 & 2 & 3 & 0 & 0\\
        0 & 0 & 1 & 2 & 3\\
        0 & 0 & 0 & 0 & 7
        \end{bmatrix}
        $
         & 
         $
        \begin{bmatrix}
        1 & -2 & 5 & 2 & -1\\
        0 & 3 & 4 & 5 & 6\\
        0 & 0 & 22 & 14 & 4
        \end{bmatrix}
        $
        \\
    \end{tabular}
\end{center}

The leading variable positions in the matrix are called \emph{pivot positions}. A column in the matrix that contains a pivot position is a \emph{pivot column}. The process of converting a linear system into echelon form is \emph{Gaussian Elimination}.

\begin{definition}[Reduced Row Echelon Form]

A matrix is said to be in \emph{reduced row echelon form} if:
\begin{enumerate}[label=\alph*)]
    \item all pivot positions have 1
    \item the only nonzero term in each pivot column is the pivot
    \item it is in row echelon form.
\end{enumerate}
\end{definition}
Try finding the reduced row echelon form of the following matrix:
\begin{equation}
    \begin{bmatrix}
    0 & 3 & 4 & 5 & 6\\
    1 & -2 & 5 & 2 & -1\\
    3 & 0 & 1 & 2 & 5
    \end{bmatrix}
\end{equation}

\begin{definition}[Homogeneity]
A \emph{homogeneous linear equation} is
\begin{equation}
    a_1x_1 + a_2x_2 + \cdots + a_nx_n = 0
\end{equation}
The equation is said to be in homogeneous form.
A linear system where all equations are in homogeneous form is a \emph{homogenous system}.
\end{definition}
Every homogenous system is \emph{consistent}, i.e. solvable.